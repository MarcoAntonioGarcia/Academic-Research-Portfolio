
\documentclass[paper,twocolumn,10pt]{article}

\usepackage[utf8]{inputenc} %teclado y acentos
\usepackage[spanish,es-tabla]{babel}
\usepackage{graphicx}         %incluir figuras
\usepackage{appendix}   
\usepackage{braket}  
\usepackage{dsfont} 
\usepackage[T1]{fontenc}
\usepackage{subcaption}     % para subfiguras
\usepackage[T1]{fontenc}
\usepackage{amsmath}
%\numberwithin{equation}{section}#numeración por capitulo
\usepackage{setspace}
\usepackage{hyperref}
\usepackage{xcolor}
\usepackage{blindtext}

\doublespacing
\onehalfspace
\singlespace
\spacing{1.25}

%----------------------------------------------------------------------------------
%MARGEN
\usepackage{vmargin}
\setpapersize{A4}
\setmargins{1.5cm}           % margen izquierdo
{0cm}                        % margen superior
{18.4cm}                     % anchura del texto
{23.42cm}                    % altura del texto
{50pt}                       % altura de los encabezados
{1cm}                        % espacio entre el texto y los encabezados
{0pt}                        % altura del pie de página
{2cm}                        % espacio entre el texto y el pie de página
%----------------------------------------------------------------------------------
%ENCABEZADOS Y PIE DE PAG
\usepackage{fancyhdr}
\pagestyle{fancy}
\fancyhf{}
\fancyhead[LO]{\leftmark} % En las páginas impares, parte izquierda del encabezado, aparecerá el nombre de capítulo
%\fancyhead[LO]{Computación Cuántica Bajo el Marco de SEAQT} 
\fancyhead[RE]{\rightmark} % En las páginas pares, parte derecha del encabezado, aparecerá el nombre de sección
\fancyhead[RO,LE]{\thepage} % Números de página en las esquinas de los encabezados
%\fancyfoot[LE,RO]{\thepage} %Escribo este texto a la izquierda en las páginas impares y a la derecha en las pares en el pie de pagina
%\renewcommand{\footrulewidth}{0.4pt} %linea de pie de pagina

%----------------------------------------------------------------------------------
%COLORES EN EL COGDIGO Y CODIGO DE PROGRAMACIÓN
\usepackage{listings,color}
\definecolor{codegreen}{rgb}{0,0.5,0}
\definecolor{codegray}{rgb}{0.5,0.5,0.5}
\definecolor{codepurple}{rgb}{0.58,0,0.82}
\definecolor{backcolour}{rgb}{0.95,0.95,0.92} 

\definecolor{backcolour}{RGB}{255,255,255}
\definecolor{codegreen}{RGB}{0,255,0}
\definecolor{blue}{RGB}{0,0,255}
\definecolor{codegray}{RGB}{100,100,100}


\lstdefinestyle{mystyle}{
    backgroundcolor=\color{backcolour},   
    commentstyle=\color{codegreen},
    keywordstyle=\color{blue},
    numberstyle=\tiny\color{codegray},
    stringstyle=\color{codepurple},
    basicstyle=\footnotesize,
    breakatwhitespace=false,         
    breaklines=true,                 
    captionpos=b,                    
    keepspaces=true,                 
    numbers=left,                    
    numbersep=5pt,                  
    showspaces=false,                
    showstringspaces=false,
    showtabs=false,                  
    tabsize=2
}

%TITULO Y AUTORES
\title{{\huge Universidad de Guanajuato}\\ 
Estudio Termodinámico de la Decoherencia en Compuertas Cuánticas Bajo La Dinámica de SEAQT\\ }
\author{García Sánchez Marco Antonio$^1$, Bautista López Juan Luis$^1$\\, Ramírez Rodríguez José de Jesús$^1$\\ Damían Ascencio César Eduardo$^1$, Cano Andrade Sergio$^1$\\ \\ \\$^1$Departamento de Ingenier\'ia Mec\'anica, Universidad de Guanajuato,\\Salamanca,Gto.\\  \\}
\date{\today}
\lstset{style=mystyle}

\setlength{\parindent}{.5cm}
\begin{document}

\twocolumn[
\begin{@twocolumnfalse}
\maketitle
\begin{abstract}
En este trabajo se muestra el estudio termodinámico de compuertas cuánticas bajo el marco de Steepest-Entropy-Ascent Quantum Thermodynamic, SEAQT por sus siglas en ingles, más específicamente se estudia la dinámica de las compuertas que componen el circuito cuántico H-CNOT, ya que este protocolo genera los estados de máximo entrelazamiento cuántico. Se presenta también la relación entre el entrelazamiento y la generación de entropía bajo en el mismo marco de referencia. \\
Palabras clave: SEAQT, H-CNOT, Concurrencia.\\
\end{abstract}
\end{@twocolumnfalse}
]

\paragraph{\normalsize{I. INTRODUCCIÓN}}\ \\

La simulación de la física siempre ha sido de gran importancia, en 1982 Richard P. Feynman, uno de los pioneros de la computación cuántica publicó un artículo titulado \textit{"Simulating Physics with Computers"} \cite{feynman1982} donde analiza las fronteras de la computación para la simulación de la física y propone la idea de usar las propiedades de la mecánica cuántica para para crear un procesador cuántico, esto más tarde formaría una rama de investigación llamada \textit{Computación Cuántica}. En \cite{feynman_computation} Feynman aborda más formalmente la idea de computación cuántica.\\
La idea principal de la computación cuántica es usar las propiedades de la mecánica cuántica y crear procesadores que tengan una gran ventaja sobre todo de rapidez comparada con los procesadores convencionales. La ventaja de la velocidad del procesamiento no podría ser igualada con la computación convencional (computación clásica) existente\cite{quantum_computation}.\\
En mecánica cuántica la ecuación de Schrödinger no predice generación de entropía (entropía de von Neumann) y es usada para la dinámica de estados puros. Hasta el momento existen pocas propuestas que describen fenómenos fuera del equilibrio sin ser por completo aceptadas, estas propuestas alternas pretenden introducir irreversibilidades al sistema . El marco de SEAQT  surge como una idea de en un sistema cerrado y propone que las irreversibilidades son intrínsecas al sistema basado en el principio de máxima producción de entropía de la mecánica estadística. Esta teoría pretende una dinámica más amplia lograda proponiendo que $\rho\geq\rho^2$ y en la construcción matemática se preserva el principio de probabilidad de la mecánica cuántica así como la conservación de la energía para Hamiltonianos independientes del tiempo \cite{wiqt}. En \cite{beretta1985} se presenta la construcción de la ecuación de movimiento que pretende describir una dinámica fuera del equilibrio y para los estados puros recuperar la ecuación de von Neumann. Este marco teórico sera utilizado en este trabajo para determinar la evolución en el circuito H-CNOT y ver la relación entre el entrelazamiento y la generación de entorpía.\\
El entrelazamiento fue introducido por Erwin Schrödinger en la década de los 30 dándole el nombre de enredo a una correlación de la naturaleza cuántica y establece que el estado cuántico de partículas entrelazadas no puede definirse individualmente, sino que la función de onda es única para el sistema, en otras palabras, el estado de ambas partículas esta determinado por la misma función de onda \cite{Caracterizando_Entrelazamiento}. Científicos como Einstein, Podolsky y Rosen lo titularon como  \textit{''Acción fantasmal a distancia''} en un artículo que contenía la paradoja EPR \cite{EPR} como un intento de desacreditar a la mecánica cuántica como una teoría completa mediante una contradicción en sus postulados \cite{Caracterizando_Entrelazamiento}. La paradoja EPR contenía una incompatibilidad con la relatividad, más específicamente con la transmisión de información superlumínica, también violaba el principio de localidad donde dos objetos se veían influenciadas incluso después de estar separadas una distancia considerable. Este pensamiento de una mecánica cuántica incompleta llevo a la comunidad científica a buscar variables ocultas las cuales complementarían la función de onda y con esto dar un carácter determinista a la teoría. Con EPR se vieron involucrados artículos científicos intentando encontrar formas de desacreditarla \cite{bohrEPR}, pero fue hasta 1964 que John Bell propuso una forma matemática de prevenir dicha paradoja \cite{bell1964}. Bell, Asumiendo que la mecánica cuántica obedece el principio de localidad y leyes deterministas logró deducir desigualdades las cuales si EPR estaba en lo cierto serian ciertas y si son violadas entonces la teoría cuántica sería una teoría completa incluso al romper con el determinismo de la física. además existiría un entrelazamiento entre los sistemas con un colapso de la función de onda transmitido instantáneamente \cite{bell1964}.\\

\paragraph{\normalsize{II. MECÁNICA QUÁNTICA}}
\subparagraph{A. Postulados}
\subparagraph{A.I. Estado;}
Un estado $\ket{\Psi}$ contiene un completa descripción del sistema, en mecánica cuántica es un vector en el espacio de Hilbert $\mathcal{H}$.

\subparagraph{A.II. Observables;}
Las cantidades físicas son representadas mediante operadores hermiticos, es decir ${T}={T}^\dagger$, donde $\dagger$ representa la operación \textit{"traspuesta conjugada"} del operador. 
Los observables están relacionados con los operadores de la forma

\begin{equation}
\label{eq:Observable}
\braket{{A}}=\sum_{n=1}\braket{\Psi_n|{\widehat{A}}|\Psi_n}
\end{equation}
donde $A$ es un promedio del observable $\widehat{A}$

\subparagraph{A.III. Mediciones;}
Los valores medidos de un operador $T$ son los eigenvalores del mismo, antes de la medición cada observable tiene una probabilidad de ser obtenido. Recordando la ecuación característica es posible escribir el observable en función de los eigenvalores del observable como

\begin{equation}
\label{eq:Observable}
\braket{{A}}=\sum_{n=1}\lambda_n\braket{\Psi_n|\Psi_n}
\end{equation}
donde $\lambda_n$ son los n eigenvalores del operador. Si el estado esta normalizado cada eigenvalor corresponde a la probabilidad de encontrar el estado asociado. Cuando se realiza una medición el función de onda colapsa en un estado en particular de todos los probables posibles \cite{Demys}

\subparagraph{A.IV. Dinámica;} La evolución de un estado cuántico es unitaria, generada por un operador llamado \textit{"Hamiltoniano"} ($H$) del sistema, y evoluciona de acuerdo con la ecuación de von Neumann

\begin{equation}
\label{eq:Ec_von_Neumann}
\frac{d \rho}{dt}=-\frac{i}{\hbar}[H,\rho].
\end{equation}

\subparagraph{B. Operador de densidad}\ \\

El formalismo de la matriz de densidad también conocido cómo operador de densidad \textit{"$\rho$"} es una representación que contienen todos los estados accesibles del sistema. Con las propiedades de este operador se puede tener una descripción total de un sistema cuántico \cite{cohen} y esta definido como

\begin{equation}
\label{eq:Densidad}
\rho=\ket{\Psi}\bra{\Psi}
\end{equation}
con la ec.(\ref{eq:Densidad}) es claro que los valores en la diagonal $\rho_{nn}$ del operador de densidad corresponden a los valores de probabilidad de encontrar el sistema en el estado $n$, los valores fuera de la diagonal $\rho_{mn}$ son conocidos como \textit{coherencias} que representan la interferencia entre los estados $\ket{\Psi_m}$ y $\ket{\Psi_n}$ \cite{sakurai}.

Los observables pueden ser calculados en función del operador de densidad como:
\begin{equation}
\label{eq:Densidad_observables}
\braket{A}=Tr(\rho_A).
\end{equation}

\subparagraph{B.I. Propiedades}\ \\
i) Hermítico $\rho=\rho^{\dagger}$\\
ii) Indempotente para estados puros $\rho=\rho^2$\\
iii) Traza unitaria $Tr(\rho)=1$\\
iv) Positivo definido $\rho\geq0$\\

Para 2 partículas, el operador de densidad se define como

\begin{equation}
\label{eq:rho2Q}
\rho=\rho A\otimes\rho_B
\end{equation}
donde $\rho_J(J=A,B)$ son los operadores de densidad de cada subsistema.



\paragraph{\normalsize{III. COMPUTACIÓN CUÁNTICA}}\ 
\subparagraph{A. Qubits}\ \\

La unidad mínima de información cuántica es el bit cuántico o qubit (quantum bit) el cual es un estado en el espacio de Hilbert ($\mathcal{H}$) de dimensión 2, en otras palabras para obtener un qubit se necesita encontrar un sistema cuántico con dos estados accesibles, este sistema puede ser representado por un electrón con una propiedad llamada spin. El spin es un concepto muy complicado de explicar pero basta con decir que puede tomar dos valores $(\frac{1}{2})$ y $(-\frac{1}{2})$ que denominaremos como los estados $\ket{0}$ y $\ket{1}$ respectivamente.\\
Los estados $\ket{0}$ y $\ket{1}$ son ortogonales y son representados como

\begin{equation}
\label{eq:cero}
\ket{0}=\begin{bmatrix}1 \\ 0 \end{bmatrix}\ \ \ \ \ket{1}=\begin{bmatrix}0 \\ 1 \end{bmatrix}.
\end{equation}

La superposición se consiste en tener los estados $\ket{0}$ y $\ket{1}$ al mismo tiempo, representando como

\begin{equation}
\label{eqn:superposición}
\ket{\Psi}=\alpha\ket{0}+\beta\ket{1}
\end{equation}
donde $|\alpha|^2=\alpha\alpha^*$ y $|\beta|^2=\beta\beta^*$ y representan la probabilidad de encontrar el qubit en el estado $\ket{0}$ o en el estado $\ket{1}$ respectivamente, entonces es evidente que $|\alpha|^2+|\beta|^2=1$, ya que el sistema solo tiene dos estados accesibles.\\
Si se tienen 2 qubits, el espacio de Hilbert del sistema esta compuesto por el de los subsistemas ($\mathcal{H}=\mathcal{H_A}\otimes\mathcal{H_B}$) y es de dimensión 4. Entonces el estado estará en una superposición de 4 posibles con probabilidades $\alpha_{xy}$ y puede ser escrito como

\begin{equation}
\label{eqn:superposición_2}
\ket{\Psi}=\alpha_{00}\ket{00}+\alpha_{01}\ket{01}+\alpha_{10}\ket{10}+\alpha_{11}\ket{11}
\end{equation}
donde $\ket{00}=\ket{0}\otimes\ket{0}$, $\ket{01}=\ket{0}\otimes\ket{1}$, $\ket{10}=\ket{1}\otimes\ket{0}$ y $\ket{11}=\ket{1}\otimes\ket{1}$.

\subparagraph{B. Compuertas cuánticas}\ \\

En computación cuántica existen compuertas que en general son operaciones unitarias sobre los qubits. Estos operadores son representados por matrices de rango igual al espacio de Hilbert que depende del número de qubits usados. Con la representación del operador de densidad para qubits así como con la esfera de Bloch será posible observar la evolución del sistema en una compuerta cuántica \cite{quantum_computation}.\\
Las compuertas cuánticas pueden ser vistas como operadores que actúan sobre un estado o qubit, estas compuertas actúan como transformaciones unitarias y por tanto cualquier compuerta cuántica tiene que ser un operador unitario el cual cumple con 	que $U^{\dagger}U=I$.\\

\subparagraph{B.I Compuerta Hadamard}\ \\

La compuerta cuántica Hadamard $U_H$ es una compuerta de un solo qubit. Es generadora de una superposición desde un estado definido, es decir, si se tiene el estado $\ket{0}$ y se le aplica una compuerta $U_h$ se generará un estado superpuesto $\ket{\Psi}$ y esta definida como

\begin{equation}
\label{eq:Hadamard}
U_H=\frac{1}{\sqrt{2}}(\sigma_x+\sigma_z)=\frac{1}{\sqrt{2}}\begin{bmatrix}1 & 1 \\ 1 & -1\end{bmatrix}.
\end{equation}

La operación de un estado sin superposición con la compuerta Hadamard genera un estado superpuesto, es decir
\begin{equation}
\label{eq:Hadamard}
U_H\ket{0}=\frac{1}{\sqrt{2}}(\ket{0}+\ket{1}).
\end{equation}

Para sistemas en con campos magnéticos como es el caso, el Hamiltoniano esta determinado por una oscilación en el eje Z 

\begin{equation}
\label{eq:Hamiltoniano}
H=-\frac{1}{2}\hbar\omega_L\sigma_z
\end{equation}
donde $\hbar$ es la constante modificada de Plank, $\omega_L$ es la frecuencia de oscilación y $\sigma_i(i=x,y,z)$ son las matrices de Pauli que junto con la matriz identidad forman una base para sistemas de dos estado (qubit).

\subparagraph{B.I Compuerta CNOT}\ \\

La compuerta \textit{"\ Controlled Not\ "} o $U_{CNOT}$, actúa sobre dos qubits, uno de ellos es llamado control y el otro objetivo y el funcionamiento esta basado en la negación del objetivo siempre y cuando el de control se encuentre en el estado $\ket{1}$ entonces se pueden tener 4 casos posibles; $U_{CNOT}\ket{00}=\ket{00}$, $U_{CNOT}\ket{01}=\ket{01}$, $U_{CNOT}\ket{10}=\ket{11}$ y $U_{CNOT}\ket{11}=\ket{10}$.
La compuerta $U_CNOT$ esta definida como

\begin{equation}
\label{eq:U_CNOT}
U_{CNOT}={\begin{bmatrix}1& 0& 0& 0\\0 & 1& 0& 0 \\0 &0 &0 &1 \\0 &0 &1 &0\end{bmatrix}}.
\end{equation}

Un Hamiltoniano para un sistema de dos qubits esta compuesto por los Hamiltonianos independientes de cada subsistema de la siguiente manera

\begin{equation}
\label{eq:H_2Q}
H_{2Q}=H_A\otimes I_B +I_A\otimes H_B 
\end{equation}
donde $H_A=H_B=-\frac{1}{2}\hbar\omega_L\sigma_z$ y son los Hamiltonianos independientes de cada subsistema.

El Hamiltoniano hace los efectos de la CNOT está matemáticamente definido como
\begin{equation}
\label{eq:H_CNOT}
H_{CNOT}=\hbar \omega (1-\theta_B(t))H_{2Q}+\hbar\theta_G(t)U_{CNOT}
\end{equation}
donde el término $\theta_B=\frac{\theta_s(t)}{\theta_s(t_0)}$ es un pulso normalizado y $\theta_S$ esta definido como

\begin{equation}
\label{eq:theta_B2}
\theta_s(t)=N_f\frac{1}{2}[Erf(\frac{t-t_1}{\tau})-Erf(\frac{t-t_2}{\tau})]
\end{equation}
el término $t_0$ es el tiempo al centro del pulso, es decir, el tiempo en el cual $\theta_B(t)$ alcanza un valor máximo de 1, $\tau$ es el ancho del pulso, en otras palabras, es el tiempo en el cual el pulso pasará desde los valores mínimo a máximo para volver a 0, $t_1$ y $t_2$ son los tiempos en los cuales se iniciará y terminará el pulso respectivamente y están definidos como; $t_1=t_0-\frac{\tau}{2}$ y $t_2=t_0+\frac{\tau}{2}$.\\
El parámetro $N_f$ queda definido como 

\begin{equation}
\label{eq:N_f}
N_f=\int_{-\infty}^{\infty}\theta_G(t) dt=\frac{\pi}{2}
\end{equation}
donde $\theta_G$ es un pulso y es calculado como
 
\begin{equation}
\label{eq:tehta_G}
\theta_G(t)=\frac{\sqrt{\pi}}{2\tau}exp[-(\frac{t-t_0}{\tau})^2]
\end{equation}

\subparagraph{C. Circuitos cuánticos}\ \\

Los circuitos cuánticos son formados por dos o mas compuertas actuando sobre qubits. Los circuitos son leídos de izquierda a derecha y cada linea representa el paso del tiempo de un qubit y sus transformaciones unitarias a través de las compuertas. Otro aspecto importante es que son acíclicos, es decir, no se permite volver desde una parte del circuito a una parte anterior, también es importante resaltar que no se permite que se unan dos cables a un tercero así como tampoco que dos cables salgan de uno solo, en otras palabras no pueden salir 2 qubits de una sola linea ya que lo prohíbe el principio de no clonación cuántica \cite{quantum_circuits}.\\

\begin{figure}[!h]
    \centering
        \includegraphics[scale=0.3]{figures/HC_NOT.JPG}  
     \caption{Circuito cuántico H-CNOT.}
     \label{fig:H-CNOT}
\end{figure}

En la Figura \ref{fig:H-CNOT} se muestra un circuito cuántico de 2 qubits. En la linea del qubit 1 se coloca una compuerta Hadamard para generar una superposición de estados y posteriormente se aplica sobre estos dos qubits una compuerta CNOT para generar un entrelazamiento ya que este circuito regresa un estado de Bell de máximo entrelazamiento.\\
Todos los circuitos cuánticos tienen como inicial el estado $\ket{0}$ y en base a este se usan las compuertas para generar el circuito deseado. 

\subparagraph{C.I. Circuito cuántico H-CNOT}\ \\
El circuito Cuántico H-CNOT es mostrado en la Figura \ref{fig:H-CNOT} y está formado por una compuerta Hadamard en el qubit de control y un qubit en el estado $\ket{0}$ en el objetivo, posteriormente operados simultáneamente por una compuerta controlled not y generar así los estados de Bell. El circuito actúa matemáticamente de la siguiente manera:\\

i) Se aplica la compuerta Hadamard $U_H$ al qubit de control.

\begin{equation}
\label{eq:Hadamard_ket0}
U_H\ket{0}=\frac{1}{\sqrt{2}}(\ket{0}+\ket{1})
\end{equation}

ii) Para que la compuerta CNOT pueda actuar sobre los dos qubits se tiene que hacer el producto tensorial entre el el quibit de control y el objetivo entonces el estado será:

\begin{equation}
\label{eq:Q_CO}
\ket{\Psi}=\frac{1}{\sqrt{2}}(\ket{00}+\ket{10})
\end{equation}

iii) Se aplica la compuerta CNOT al estado anterior, recordando los efectos de esta ultima compuerta, se niega el estado del qubit objetivo si el de control esta en el estado 1, entonces se obtiene después del circuito el estado:

\begin{equation}
\label{eq:estado_CNOT}
U_{CNOT}\ket{\Psi}=\frac{1}{\sqrt{2}}(\ket{00}+\ket{11})
\end{equation}

El estado anterior es un estado de Bell \cite{Caracterizando_Entrelazamiento} con el más alto entrelazamiento posible. 
En términos del operador de Densidad ($\rho=\rho_A\otimes\rho_B$) para determinar la evolución del estado con la ecuación de movimiento de SEAQT para dos qubits Ec. (\ref{eqn:SEAQT_2particulas}), el operador de densidad  queda escrito como

\begin{equation}
\label{eq:RHO_inicial}
\rho={\begin{bmatrix}\frac{1}{2}& 0& \frac{1}{2}& 0\\0 & 0& 0& 0 \\\frac{1}{2} &0 &\frac{1}{2} &0 \\0 &0 &0 &0\end{bmatrix}}
\end{equation}
de esta ecuación se pueden determinar la concurrencia y se llega a un valor de 1 y entonces por eso se dice que genera un estado de entrelazamiento máximo.

\paragraph{\normalsize{IV. MODELO MATEMÁTICO}}\ 
\subparagraph{A. SEAQT}\ \\
La ecuación de movimiento en el marco de SEAQT está definida en \cite{beretta1985}, para dos partículas  puede ser escrita como

\begin{equation}
\label{eqn:SEAQT_2particulas}
\frac{d\rho}{dt}=\frac{-i}{\hbar}[H,\rho]-(\frac{1}{\tau_{D_A}}D_A\otimes\rho_B+\frac{1}{\tau_{D_B}}D_B\otimes\rho_A)
\end{equation}
donde $\tau_{D_J}$ es un parámetro libre que en \cite{wiqt} se menciona como un parámetro no caracterizado y que es de los principales temas de investigación para los principales autores de este marco de referencia, $D_J$ es el término de disipación y esta definido como

\begin{equation}
\label{eqn:SEAQT_2particulas2}
D_J=\frac{1}{2}(\sqrt{\rho}\widetilde{D}_J+(\sqrt{\rho}\widetilde{D}_J)^\dagger)
\end{equation}
la operación $\dagger$ significa transpuesto conjugado y el término $\widetilde{D}_J$ se puede calcular como

\begin{equation}
\label{eqn:SEAQT_2particulas2determinate}
\widetilde{D}=\frac{\begin{bmatrix}
    \sqrt{\rho}\ln(\rho) & \sqrt{\rho}I & \sqrt{\rho}H  \\
    (I,\ln(\rho) &(I,I) & (I,H)   \\
     (H,\ln(\rho) & (H,I) & (H,H)
\end{bmatrix}}{\begin{bmatrix}
  (I,I) & (I,H)\\
  (H,I) & (H,H)
\end{bmatrix}}
\end{equation}
donde $(A,B)$ es el producto interno de Hilbert-Schmidt definido en el espacio de Hilbert $\mathcal{H}$. los operadores de densidad de cada subsistema pueden ser determinados mediante las trazas parciales.

\subparagraph{B. Caracterización del entrelazamiento}\ \\

Una medida del entrelazamiento de estados mixtos usada hasta el momento ha sido explicada en bastantes artículos y por diferentes autores, en \cite{Caracterizando_Entrelazamiento} es nuevamente explicada y de define el entrelazamiento como

\begin{equation}
\label{eq:entrelazamiento_formacion_mixto}
E(\rho)=\epsilon\left(\frac{1+\sqrt{1+C(\rho)^2}}{2}\right)
\end{equation}
donde

\begin{equation}
\label{eq:epsilon}
\begin{split} 
\epsilon=&-\frac{1+\sqrt{1-C^2}}{2}Log_2\left(\frac{1+\sqrt{1+C(\rho)^2}}{2}\right)\\&-\frac{1-\sqrt{1-C^2}}{2}Log_2\left(\frac{1-\sqrt{1+C(\rho)^2}}{2}\right)
\end{split} 
\end{equation}
donde $C$ es la concurrencia, ademas ya que $\epsilon$ tiene un valor entre 0 y 1 a medida que C va de 0 a 1 se puede tomar el valor de la concurrencia como una medida del entrelazamiento.\\
Para un estado de 2 qubits la concurrencia es medida como

\begin{equation}
\label{eq:concurrencia}
C(\rho)=max(0,\lambda_4-\lambda_3-\lambda_2-\lambda_1)
\end{equation}
donde $\lambda_i$ son los valores propios ordenados de menor a mayor de la matriz $R=\sqrt{\sqrt{\rho}\tilde{\rho}\sqrt{\rho}}$, esta matriz representa el entrelazamiento en función del operador de densidad y mide que tan cerca están los estados de los estados generalizados de Bell \cite{harmony,entanglement_of_formation_2Q}.

\paragraph{\normalsize{IV. RESULTADOS Y DISCUSIÓN}}
\subparagraph{A. Polarización y Esfera de Bloch}\ \\

Para los resultados se hizo un programa con el cual fueron reproducidos los resultados de \cite{Carlos_M} donde se había estudiado la compuerta CNOT bajo el mismo marco y esto sirve como validación. Los parámetros usados para la obtención de resultados de esta sección son presentados en la Tabla \ref{tab:Parámetros para simulación del circuito H-CNOT}.

En la Figura \ref{fig:Polarización} se muestra como evoluciona el vector de polarización en en la esfera de Bloch, cuando el tiempo de evolución llega al $t_1$ la compuerta se activa y empieza el cambio en el vector de polarización. En la Figura \ref{fig:Bloch} se muestra que el qubit objetivo inicia su evolución en el estado $\ket{0}$. Es importante mencionar que esta figura solo muestra los efectos de la compuerta cuántica CNOT pero los qubits iniciales (control y objetivo) ya fueron analizados en el marco de SEAQT y evolucionados con los parámetros de la Tabla \ref{tab:Parámetros para simulación del circuito H-CNOT}. Esto se puede notar con los vectores naranja y verde en la figura que apuntan a los estados iniciales con los que se se generará el operador de densidad de acuerdo con la Ec. (\ref{eq:rho2Q}), ya que si la compuerta Hadamard fuera evolucionada desde un estado muy próximo a $\ket{0}$ el vector tendría que apuntar a la superficie de la esfera exactamente a la mitad entre el estado $\ket{0}$ y $\ket{1}$ representando la superposición. Que el vector apunte dentro de la esfera es un comportamiento típico del marco de SEAQT donde el estado esta generando evolucionando a un estado mixto.

\begin{table}[!ht]   %para crear el pie de tabla.
\caption{Parámetros para simulación del circuito H-CNOT}
\centering
\begin{tabular}{cccc}   %ll para alinear, cc para centrar primera columna y segunda
\hline       %linea horizontal
\textbf{Parámetro}  & \textbf{Valor} \\
\hline
\textbf{Parámetros Generales} & &\\
$\omega_L$  & 0.2675 GHz \\
$\hbar$ & 6.62x$10^{-34}$ J s\\
$k_B$  &  1.38064852x$10^{-23}$ J/K \\
& &\\
\textbf{Compuerta Hadamard} &\\
$P_x$  & 0.001 \\
$P_y$ &  0.0 \\
$P_z$ &  0.999 \\
$t_s$=tiempo de simulación & 40 $ns$ \\
$\tau_D$ & 40 $ns$ \\
$\tau$ &  5 $ns$  \\
$t_0$ & 10 $ns$\\
& &\\
\textbf{Qubit objetivo} &\\
$P_x$  & 0.001 \\
$P_y$ &  0.0 \\
$P_z$ &  0.999 \\
$t_s$=tiempo de simulación & 40 $ns$ \\
$\tau_D$ &  40 $ns$ \\
& &\\
\textbf{Compuerta CNOT} &\\
$t_s$=tiempo de simulación & 180 $ns$ \\
$\tau_D$ & 40 $ns$ \\
$\tau$ &  5 $ns$  \\
$t_0$ & 10 $ns$\\
$\Gamma$ & 0.0002  \\
\hline     
\end{tabular} 
\label{tab:Parámetros para simulación del circuito H-CNOT}
\end{table}

\begin{figure}[!h]
    \centering
        \includegraphics[scale=0.6]{figures/PolarizacionB.pdf}  
     \caption{Evolución del vector de polarización en el protocolo H-CNOT bajo el marco de SEAQT.}
     \label{fig:Polarización}
\end{figure}


\begin{figure}[!h]
    \centering
        \includegraphics[scale=0.6]{figures/Grafica2bloch2.pdf}  
     \caption{Representación de la evolución del vector de polarización en la esfera de Bloch del protocolo H-CNOT en el marco de SEAQT}
     \label{fig:Bloch}
\end{figure}


\subparagraph{A. Entropía}\ \\

De acuerdo con la entropía de von Neumann:

\begin{equation}
\label{eq:entropia}
S=-k_BTr(\rho Ln(\rho))
\end{equation}
se puede determinar la entropía total del sistema en general. En la Figura \ref{fig:Entropia} se muestra como el marco de SEAQT muestra un aumento pronunciado de la entropía del sistema en el tiempo que es activada la compuerta, esto es corroborado por una taza de generación de la misma propiedad en la Figura \ref{fig:GeneracionEntropia} que muestra un alto valor que aumenta de forma de pulso en el tiempo que la compuerta se activa. todo esto nos permite afirmar que el marco de SEAQT predice alta generación de entropía que podría ser asociada a la compuerta cuántica.\\

\begin{figure}[!h]
    \centering
        \includegraphics[scale=0.6]{figures/Grafica_entropy_total.pdf}  
     \caption{Entropía total del sistema para cada instante de la evolución bajo el marco de SEAQT.}
     \label{fig:Entropia}
\end{figure}


\begin{figure}[!h]
    \centering
        \includegraphics[scale=0.6]{figures/Generacion_Entropia.pdf}  
     \caption{Generación de entropía total del sistema evolucionando de acuerdo con el marco de SEAQT.}
     \label{fig:GeneracionEntropia}
\end{figure}


\subparagraph{A. Concurrencia}\ \\

La concurrencia ya había sido definida como una medida del entrelazamiento para estados mixtos, entonces es de vital importancia determinar que tan entrelazados pueden estar dos qubits en el marco de SEAQT.
En la Figura \ref{fig:C} se muestra como el circuito H-CNOT presenta una concurrencia que aumenta desde $0$ muy rápido al mismo tiempo que la compuerta es activada y por tanto el entrelazamiento aumenta. Esto concuerda con la física del problema ya que el sistema antes de la compuerta CNOT cuenta con subsistemas no correlacionados. Un valor positivo de concurrencia indica que hay entrelazamiento. En la Figura \ref{fig:CT} se observa como al modificar el parámetro $\tau_{D_J}(J=A,B)$ de la compuerta CNOT y manteniendo en $\tau_{D_H}=\tau_{D_{QO}}=40 \nu s$ (para generar un estado muy cercano al la superposición ideal de la compuerta Hadamard) se obtiene una pérdida de concurrencia mientras el $\tau_{D_J}(J=A,B)$ aumenta. De acuerdo con la Ec. (\ref{eqn:SEAQT_2particulas}) se puede ver que mientras el valor de $\tau_{D_J}(J=A,B)\to \infty$ se recupera la ecuación de von Neumann y se genera el caso sin disipación. 

\begin{figure}[!h]
    \centering
        \includegraphics[scale=0.6]{figures/Concurrencia.pdf}  
     \caption{Concurrencia generada por el protocolo H-CNOT en el marco de SEAQT.}
     \label{fig:C}
\end{figure}

\begin{figure}[!h]
    \centering
        \includegraphics[scale=0.6]{figures/ConcurrenciaT.pdf}  
     \caption{Concurrencia generada por el protocolo C-NOT en la dinámica de SEAQT para diferentes valores de $\tau_{D_{A,B}}$ manteniendo constante el parámetro $\tau_{D_{H,QO}}$.}
     \label{fig:CT}
\end{figure}

La Figura \ref{fig:entropyCC} muestra que mientras aumenta el termino $\tau_{D_J}(J=A,B)$ la entropía total del sistema disminuye, entonces con las discusiones anteriores se puede decir que el poco entrelazamiento generado entre subsistemas se debe a la alta producción de entropía y podemos relacionar esta ultima con la no generación de coherencias en el operador de densidad.

\begin{figure}[!h]
    \centering
        \includegraphics[scale=0.6]{figures/Grafica_entropy_subsistemasP.pdf}  
     \caption{Entropía total del sistema bajo la evolución del mismo en el protocolo H-CNOT en el marco de SEAQT para diferentes valores de $\tau_{D_{A,B}}$ manteniendo constante el parámetro $\tau_{D_{H,QO}}$.}
     \label{fig:entropyCC}
\end{figure}

Finalmente en la Figura \ref{fig:CT2} se ve como al mantener el tiempo $\tau_{D_J}(J=A,B)=40ns$ y variando $\tau_{D_K}(K=H,QO)$, mientras mas pequeño sea, la generación concurrencia disminuye de una manera muy severa, esto indica que la principal pérdida de concurrencia esta asociada a la compuerta Hadamard ya que aunque el estado generado por la misma es una superposición 50-50, las irreversibilidades se hacen presentes disminuyendo los términos fuera de la diagonal del estado y aunque teóricamente generaría un estado de Bell de máximo entrelazamiento, los qubits tendrían un valor de concurrencia pequeño lo cual es perjudicial para la concurrencia. 

\begin{figure}[!h]
    \centering
        \includegraphics[scale=0.6]{figures/ConcurrenciaT2.pdf}  
     \caption{Concurrencia generada por el protocolo C-NOT en la dinámica de SEAQT para diferentes valores de $\tau_{D_{H,QO}}$ manteniendo constante el parámetro $\tau_{D_{A,B}}$.}
     \label{fig:CT2}
\end{figure}

\paragraph{\normalsize{IV. CONCLUSIONES }}\ \\
La finalidad de este trabajo fue relacionar la decoherencia utilizando la concurrencia en sistemas de estados mixtos con la entropía bajo el marco de SEAQT, se concluye que como se esperaba, el término de disipación genera irreversibilidades en el sistema las cuales impiden la generación de entrelazamiento, mientras se tenga un aumento en la disipación se obtendrá un menor valor de concurrencia.

El parámetro $\tau_{D_J}(J=A,B)$ es entonces un factor que se puede asociar con un término en el cual el estado alcanza un nuevo equilibrio, dicho tiempo esta aun si caracterizar y sería un gran aporte su caracterización ya que se podría determinar o aproximar el tiempo en el cual el sistema pierde la coherencia e incluso intentar controlarla.

Finalmente se concluye que las irreversibilidades en el qubit con compuerta Hadamard son importante ya que mientras el estado tiende más hacia el centro de la esfera de Bloch de generan valores bajos de concurrencia, esto ya que este es tomado para ser el qubit de control de la compuerta CNOT. Entonces se concluiría que son mas importantes las irreversibilidades en la compuerta Hadamard y es necesario el estudio de dicha compuerta en el marco de SEAQT.

\section{Referencias}
\bibliographystyle{ieeetr}
\bibliography{Referencias.bib}

\end{document}
